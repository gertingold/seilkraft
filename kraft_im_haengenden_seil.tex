\documentclass[twocolumn,aps]{revtex4}
\usepackage[utf8]{inputenc}
\usepackage[T1]{fontenc}
\usepackage{lmodern}
\usepackage[ngerman]{babel}
\usepackage{graphicx}
\graphicspath{{./img/}}
\begin{document}
\title{Kräfte im hängenden Seil}
\author{Gert-Ludwig Ingold}
\affiliation{Institut für Physik, Universität Augsburg, 86135 Augsburg}
\begin{abstract}
 In diesen Notizen soll die maximale Kraft hergeleitet werden, die in
 einem undehnbaren hängenden Seil auftritt. Dazu wird zunächst
 mit Hilfe der Betrachtung der auf ein kurzes Seilstück
 wirkenden Kräfte der Seilverlauf, die sogenannte Kettenlinie,
 hergeleitet. Anschließend lässt sich die maximale Seilkraft
 bestimmen.
\end{abstract}
\maketitle

\section{Kräftegleichgewicht an einem Seilstück}
Zunächst wird der Seilverlauf eines undehnbaren, im Gravitationsfeld
hängenden Seils, die sogenannte Kettenlinie, berechnet. Es handelt
sich dabei um ein Variationsproblem mit Nebenbedingungen, in dem die
potentielle Energie des Seils bei vorgegebener Seillänge $L$ minimiert
werden soll. Dies kann mit Hilfe von Lagrange-Multiplikatoren und der
zugehörigen Euler-Lagrange-Gleichung erfolgen. Da wir uns aber in
erster Linie für die Seilkräfte interessieren, ist es angebracht, die
Kettenlinie stattdessen durch eine Kräftebetrachtung herzuleiten.

Wir beginnen mit der Betrachtung der in Abb.~\ref{fig:kraefte}
dargestellten Kräfte auf ein kurzes Seilstück der Länge $\mathrm{d}s$,
das sich am Ort $x_0$ befindet. Auf dieses Seilstück mit der
längenbezogenen Dichte $\mu$ wirkt die Gewichtskraft, die in die
negative $y$-Richtung zeigt und den Betrag
\begin{equation}
 G = \mu g\mathrm{d}s
\end{equation}
besitzt. Dabei ist $g$ die Erdbeschleunigung. Zudem üben die links und
rechts benachbarten Seilstücke die Kräfte $\vec F^{(-)}$ bzw.\ $\vec
F^{(+)}$ aus. Da das Seil keine Querkräfte zulassen soll, wirken diese
Kräfte in Richtung der jeweiligen Seilstücke. Im Gleichgewicht gilt
somit
\begin{equation}
 \vec F^{(+)}+\vec F^{(-)}+\vec G = 0\,.
 \label{eq:kraftgleichgewicht}
\end{equation}

\begin{figure}
 \includegraphics[width=0.8\columnwidth]{kraefte}
 \caption{Auf ein kurzes Seilstück der Länge $\mathrm{d}s$ wirken
	  die Gewichtskraft $\vec G$ sowie die Seilkräfte $\vec F^{(-)}$
	  und $\vec F^{(+)}$ von links bzw.\ von rechts. Die grau
	  dargestellten Vektoren zerlegen die Seilkräfte in die
	  horizontalen und vertikalen Komponenten.}
 \label{fig:kraefte}
\end{figure}

Diese Gleichgewichtsbedingung wird nun in ihre horizontalen und
vertikalen Komponenten zerlegt, wobei die Zerlegung der Seilkräfte in
Abb.~\ref{fig:kraefte} durch die grauen Vektoren dargestellt ist. Da
die Gewichtskraft keine horizontale Komponente besitzt, folgt
\begin{equation}
 F^{(+)}_x = -F^{(-)}_x = \mu g a\,.
 \label{eq:kraftgleichgewicht_x}
\end{equation}
Diese Kraftkomponente ist entlang des gesamten Seils konstant und wird
von der Aufhängung aufgebracht. Es ist günstig, die horizontale
Kraftkomponente in der angegebenen Form zu schreiben, wobei $a$ eine
Konstante mit der Dimension einer Länge ist, deren Bedeutung später
noch klar werden wird.

Beschreibt man die Seilkurve durch eine noch zu bestimmende Funktion
$y(x)$, so folgt aus der Tatsache, dass die Seilkräfte nur entlang des
Seils wirken können, dass das Verhältnis der Seilkraftkomponenten
gemäß
\begin{equation}
 \frac{F_y}{F_x} = \frac{\mathrm{d}y}{\mathrm{d}x} = y'(x)\,,
\end{equation}
durch die Ableitung der Seilkurve gegeben ist. Zur Vereinfachung der
Notation soll ein Strich die Ableitung nach der Koordinate $x$
bedeuten. Aus (\ref{eq:kraftgleichgewicht}) erhält man somit zusammen
mit (\ref{eq:kraftgleichgewicht_x}) für die vertikalen
Kraftkomponenten
\begin{equation}
 \mu ga\left[y'(x_0+\mathrm{d}x/2)-y'(x_0-\mathrm{d}x/2)\right]-\mu
 g\mathrm{d}s = 0\,.
\end{equation}
Hier sind wir von dem bisher verwendeten speziellen Ort $x_0$ des
Seilstücks zu einem beliebigen Ort $x$ übergegangen.

Dividiert man durch $\mathrm{d}x$, so wird aus der Differenz der
Ableitungen von $y(x)$ eine zweite Ableitung
\begin{equation}
 ay''(x)-\frac{\mathrm{d}s}{\mathrm{d}x} = 0\,.
 \label{eq:zweite_ableitung}
\end{equation}
Den Zusammenhang zwischen den Differentialen erhält man mit Hilfe des
Satzes von Pythagoras, der auf unser Problem angewandt als
\begin{equation}
 \mathrm{d}s^2 = \mathrm{d}x^2+\mathrm{d}y^2
\end{equation}
geschrieben werden kann. Daraus folgt
\begin{equation}
 \mathrm{d}s = \sqrt{1+y'(x)^2}\mathrm{d}x\,,
\end{equation}
so dass wir mit (\ref{eq:zweite_ableitung}) die Gleichung einer
Kettenlinie
\begin{equation}
 ay''(x)-\sqrt{1+y'(x)^2} = 0
\end{equation}
erhalten.

\end{document}
